\documentclass[12pt]{article}
\usepackage{luatexja}
\usepackage{luatexja-fontspec}
\usepackage{fontspec}
\usepackage[margin=28mm]{geometry}
\usepackage{microtype}
\usepackage{setspace}
\setstretch{1.08}
\setmainfont{Libertinus Serif}
\setmainjfont{HaranoAjiMincho}

\begin{document}
\pagestyle{empty}
{\normalsize\bfseries Keisuke Hoshino}\par
\small
Research Institute for Mathematical Sciences,
Kyoto University, Kyoto 606-8502, Japan\par
hoshinok@kurims.kyoto-u.ac.jp
\quad +81-80-5456-5530
\bigskip\hrule\bigskip
\bigskip

\noindent\textbf{Subject:} Application for Research Associate - Category Theory and Machine Learning (Fixed Term)\par
\medskip
\noindent Dear Selection Committee,
\medskip

% ---(以下、上の本文をそのまま貼り付け)---

I am Keisuke Hoshino, a PhD student in Mathematics at Kyoto University, and I expect to complete my PhD in February 2026.
While the advertised start date is 1 October 2025 (or as soon as possible thereafter), the scheduling of my PhD completion means that I would be available to commence in post from February 2026.
I would be grateful if this timing could be accommodated; I would, of course, be happy to discuss arrangements that suit the project.

My research lies in (higher) category theory. I have authored several papers on double categories and weak $\omega$-categories, and my doctoral thesis develops a homotopy theory for Batanin-Leinster style weak $\omega$-categories. More broadly, I read and work across several areas of categorical algebra and higher category theory. This background shows that I have strong expertise in areas listed in (1).

Regarding requirement (2) (Machine Learning), I have recently joined a project with Takahiro Sanada, Kenshin Hirai, and Shinya Katsumata on a programming language for building architectures of ML, using the idea of promonads and the perspective of algebraic effects. I am responsible for the categorical semantics. Some partial results were presented (in Japanese) at the 42nd Annual Conference of the Japan Society for Software Science and Technology (JSSST) under the title ``Towards a Neural Network Programming Language Based on Reverse Differential Categories and Arrow Handlers'' (Japanese: 逆微分圏に基づいたアローハンドラによるニューラルネットワークプログラミング言語に向けて).
This demonstrates not only my commitment to this direction, but also my ability to connect ML-motivated questions with precise categorical frameworks.

I believe I would be an excellent fit for Professor Jamie Vicary's group. Professor Vicary has extensive work on weak $\omega$-categories, and the group includes researchers in algebraic higher category theory. My specialisation in weak $\omega$-categories would allow me to contribute immediately, while my current work on an ML programming language positions me to collaborate on category-theoretic foundations for ML.

The Department's international reputation and collaborative ethos make it an ideal home for my research at the interface of higher category theory and ML semantics.
I aim to contribute fundamental results on weak $\omega$-categories and to work with colleagues to realise semantics-informed ML methods with impact beyond the University.

\medskip
\noindent Yours sincerely,
\par
\vspace{.6\baselineskip}
Keisuke Hoshino
\end{document}
