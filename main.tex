\documentclass[a4paper,dvipsnames,11pt,backend=luatex]{article} %lualatex
\usepackage{luatexja}

% 余白と基本体裁
\usepackage[left=15mm,right=15mm,top=15mm,bottom=15mm]{geometry}
\usepackage{enumitem}       % 箇条書きの余白調整
\usepackage[colorlinks]{hyperref}
\hypersetup{%
urlcolor=black,    % URL の色
citecolor=blue,   % 引用の色
}
\usepackage{tabularx}        % 自動幅調整
\setlength{\tabcolsep}{3pt}  % 列間余白をぎりぎりまで詰める
\renewcommand{\arraystretch}{1.6} % 行送り少しゆったり

% ─────────── Biblatex 設定 ───────────
\usepackage[
  backend=biber,
  style=numeric,        % [1] Soichiro Fujii ... の感じ
  sorting=ydnt,         % 年 ↓, 著者昇順
  maxbibnames=99,       % 省略しない
  giveninits=true,       % イニシャル表記を抑止
  defernumbers=true,      % 引用順に並べる
]{biblatex}
\addbibresource{bibliography.bib}  % ←自分の .bib

% DOI/URL を最後に付けたいだけなら標準マクロで充分
\AtEveryBibitem{\clearname{editor}}  % editor は消す例

% ─────────── CV 用マクロ (任意) ───────────
\newcommand*{\cvsection}[1]{%
  \vspace{1.2\baselineskip}% 章間の余白(好みで調整)
  {\raggedright          % flushleft(折り返しは右端を揃えない)
   \bfseries\large       % 太字+\large = \normalsize より一回り大
   #1\par}               % 見出しテキスト
  \vspace{0.4\baselineskip}%
}
\setlist[itemize]{leftmargin=*,label={},topsep=0pt,parsep=0pt}

\begin{document}




\begin{center}\large\bfseries Curriculum Vitae\footnote{Last updated: \today}
\\
\rm Keisuke Hoshino
\end{center}
\noindent %
\cvsection{Contact details}
\vspace{-1em}%
\noindent %
Division of Mathematics and Mathematical Sciences, Faculty of Science, Kyoto University,
\\
\href{mailto:hosiksk215@gmail.com}{hosiksk215@gmail.com}


\cvsection{Education}
\begin{tabularx}{\textwidth}{@{}lp{140mm}}
  2023--
  &
  Ph.D.\ course in Division of Mathematics and Mathematical Sciences, Faculty of Science, Kyoto University.
  \\
  2021--2023
  &
  Master's course in Division of Mathematics and Mathematical Sciences, Faculty of Science, Kyoto University.
  \\&
  Thesis title : \textit{Towards structures of higher dimensional categories}
  \\
  2017--2021
  &
  Undergraduate course in Faculty of Science, Kyoto University.
\end{tabularx}

\cvsection{Publications}
\begin{refsection}                      % ← 他所で引用しても干渉しない
  \nocite{*} % ← すべての文献を表示
  \printbibliography[heading=none,keyword=published,resetnumbers=true]
\end{refsection}

\cvsection{Preprints}
\begin{refsection}
\nocite{*} % ← すべての文献を表示
  \printbibliography[heading=none,keyword=preprint,resetnumbers=true]
\end{refsection}

\cvsection{Talks}
(Invited)
\begin{refsection}
\nocite{*} % ← すべての文献を表示
  \printbibliography[heading=none,keyword=iv-talk,resetnumbers=true]
\end{refsection}

(non-refereed)
\begin{refsection}
\nocite{*} % ← すべての文献を表示
  \printbibliography[heading=none,keyword=nr-talk,resetnumbers=true]
\end{refsection}

\pagebreak
\cvsection{Teaching experiences}
\begin{tabularx}{\textwidth}{@{}lp{140mm}}
  2021
  &
  Teaching assistant in Faculty of Science, Kyoto University.
name: LaTeX compilation
on:
  push:
    branches:
      - main
jobs:
  build:
    runs-on: ubuntu-latest
    steps:
      - name: Set up Git repository
        uses: actions/checkout@v4
        with:
          fetch-depth: 0
      - name: Merge main branch
        run: |
          git config user.name HoshinoKeisuke
          git config user.email catkekke18635@gmail.com
          git checkout pdf
          git merge main
      - name: Compile LaTeX document
        uses: xu-cheng/latex-action@v3
        with:
          root_file:
            main.tex
          latexmk_use_lualatex: true
      - name: Push PDF file
        run: |
          git add --all
          git commit -m "LaTeX compilation"
          git push --force origin pdf

# In the repository, go Setting/Actions/General/Workflow permissions and set to "read and write perission"
  Instructor for KTGU Faculty Seminar reading \textit{Introduction to the Theory of Computation} by Sipser, Michael.
  \\
  2022
  &
  Teaching assistant in Faculty of Science, Kyoto University.
  Instructor for KTGU Faculty Seminar reading \textit{プログラム意味論} by Yokouchi, Hirofumi.
\end{tabularx}

\end{document}

