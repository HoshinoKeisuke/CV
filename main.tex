\documentclass[a4paper,dvipsnames,11pt,backend=luatex]{article} %lualatex
\usepackage{luatexja}

% 余白と基本体裁
\usepackage[left=15mm,right=15mm,top=15mm,bottom=15mm]{geometry}
\usepackage{enumitem}       % 箇条書きの余白調整
\usepackage[colorlinks]{hyperref}
\hypersetup{%
urlcolor=black,    % URL の色
citecolor=blue,   % 引用の色
}
\usepackage{tabularx}        % 自動幅調整
\setlength{\tabcolsep}{3pt}  % 列間余白
\renewcommand{\arraystretch}{1.6} % 行送り

% ─────────── Biblatex 設定 ───────────
\usepackage[
  backend=biber,
  style=numeric,        % [1] Soichiro Fujii ... の感じ
  sorting=ydnt,         % 年 ↓, 著者昇順
  maxbibnames=99,       % 省略しない
  giveninits=true,       % イニシャル表記を抑止
  defernumbers=true,      % 引用順に並べる
]{biblatex}
\addbibresource{bibliography.bib}  % ←自分の .bib

% DOI/URL を最後に付けたいだけなら標準マクロで充分
\AtEveryBibitem{\clearname{editor}}  % editor は消す例

% ─────────── CV 用マクロ (任意) ───────────
\newcommand*{\cvsection}[1]{%
  \vspace{1.2\baselineskip}% 章間の余白(好みで調整)
  {\raggedright          % flushleft(折り返しは右端を揃えない)
   \bfseries\large       % 太字+\large = \normalsize より一回り大
   #1\par}               % 見出しテキスト
  \vspace{0.4\baselineskip}%
}
\setlist[itemize]{leftmargin=*,label={},topsep=0pt,parsep=0pt}

\begin{document}




\begin{center}\large\bfseries Curriculum Vitae\footnote{Last updated: \today}
\\
\rm Keisuke Hoshino
\end{center}
\noindent %
\cvsection{Contact details}
\vspace{-1em}%
\noindent %
Division of Mathematics and Mathematical Sciences, Faculty of Science, Kyoto University,
\\
Kitashirakawa Oiwake-chō, Sakyō-ku, Kyoto-shi,
Kyoto 606-8502  
Japan
\\
Email 1: \href{mailto:hosiksk215@gmail.com}{hosiksk215@gmail.com}
\\
Email 2: \href{mailto:hoshinok@kurims.kyoto-u.ac.jp}{hoshinok@kurims.kyoto-u.ac.jp}
\\
Phone: +81 80-5456-5530


\cvsection{Education}
\begin{tabularx}{\textwidth}{@{}lp{140mm}}
  2023--
  &
  Ph.D.\ course in Division of Mathematics and Mathematical Sciences, Faculty of Science, Kyoto University.
  \\
  2021--2023
  &
  Master's course in Division of Mathematics and Mathematical Sciences, Faculty of Science, Kyoto University.
  \\&
  Thesis title : \textit{Towards structures of higher dimensional categories}
  \\
  2017--2021
  &
  Undergraduate course in Faculty of Science, Kyoto University.
\end{tabularx}

\cvsection{Publications}
\begin{refsection}                      % ← 他所で引用しても干渉しない
  \nocite{*} % ← すべての文献を表示
  \printbibliography[heading=none,keyword=published,resetnumbers=true]
\end{refsection}

\cvsection{Preprints}
\begin{refsection}
\nocite{*} % ← すべての文献を表示
  \printbibliography[heading=none,keyword=preprint,resetnumbers=true]
\end{refsection}

\cvsection{Talks}
(Invited)
\begin{refsection}
\nocite{*} % ← すべての文献を表示
  \printbibliography[heading=none,keyword=iv-talk,resetnumbers=true]
\end{refsection}

(non-refereed)
\begin{refsection}
\nocite{*} % ← すべての文献を表示
  \printbibliography[heading=none,keyword=nr-talk,resetnumbers=true]
\end{refsection}


\cvsection{Teaching experiences}
\begin{tabularx}{\textwidth}{@{}lp{140mm}}
  2021
  &
  Teaching assistant in Faculty of Science, Kyoto University.
  Tutor for KTGU Faculty Seminar (KTGU学部セミナー) reading \textit{Introduction to the Theory of Computation} by Sipser, Michael.
  \\
  2022
  &
  Teaching assistant in Faculty of Science, Kyoto University.
  Tutor for KTGU Faculty Seminar (KTGU学部セミナー) reading \textit{プログラム意味論 (Program Semantics) } by Yokouchi, Hirofumi.
\end{tabularx}

\end{document}

